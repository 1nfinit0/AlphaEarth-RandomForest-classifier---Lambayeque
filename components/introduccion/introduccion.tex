\vspace*{\fill}
\section{\texorpdfstring{\centering INTRODUCCIÓN}{INTRODUCCIÓN}}

  % Se contextualiza desde inicios de la agrcultura a la actualidad con el avances tecnológicos mu avanzado como son los satélites.

  Para tratar del uso de información satelital en la agricultura de precisión, es importante entender primero el contexto histórico y tecnológico que ha llevado a su desarrollo. La agricultura ha sido una actividad fundamental para la humanidad desde hace aproximadamente 12,000 años, cuando la mayoría de las poblaciones humanas pasó de la recolección a diversas formas de cultivo. \parencite{Wells2020} Hasta ese momento, las sociedades humanas dependían de la caza y la recolección para su subsistencia. La transición a la agricultura permitió el desarrollo de asentamientos permanentes y el crecimiento de las civilizaciones. Desde entonces, la agricultura ha evolucionado significativamente, pasando de técnicas rudimentarias a métodos más sofisticados y eficientes. La Revolución Agrícola, que tuvo lugar entre los siglos XVIII y XIX, marcó un punto de inflexión en la historia de esta práctica. Durante este período, se introdujeron nuevas tecnologías y prácticas agrícolas, como la rotación de cultivos, el uso de fertilizantes, la mecanización de las labores agrícolas o la sintetización de productos químicos basados en nitrógeno. Estos avances permitieron un aumento significativo en la productividad agrícola y sentaron las bases para la agricultura moderna.

  Sin embargo, el desarrollo de mejores prácticas agrícolas que puedan optimizar el uso de recursos, mejorar el rendimiento y mitigar los impactos negativos de factores como las enfermedades, plagas, entre otros, ha sido un desafío constante para los agricultores a lo largo de dicho tiempo. El poder identificar estos problemas de manera temprana y precisa, además de convertirse en una ventaja competitiva en el mercado actual, se ha vuelto una necesidad imperante para garantizar la seguridad alimentaria pero llega a ser una labor compleja y más, cuando se trata de grandes extensiones de terreno del orden de miles de hectáreas. En este contexto, la agricultura de precisión ha surgido como una solución innovadora, creativa y altamente tecnológica para abordar estos desafíos. La agricultura de precisión, según \parencite{Vullaganti2025}, es una práctica que utiliza tecnologías avanzadas y enfoques basados en datos para ayudar en la toma de decisiones y optimizar la producción de cultivos. Dentro de estas tecnologías, la teledetección satelital ha ganado una importancia significativa debido a su capacidad para proporcionar información detallada y en tiempo real sobre las condiciones del suelo, el estado de los cultivos y otros factores relevantes para la gestión agrícola.

  La teledetección satelital es el producto de la convergencia de muchas disciplinas científicas y tecnológicas que incluyen a la física, la ingeniería, la informática y las ciencias de la tierra. Desde Newton y su estudio sobre la luz, óptica, el espectro electromagnético y gravitación universal hasta la invención de los primeros satélites artificiales en la década de 1950, la humanidad ha avanzado significativamente en su capacidad para observar y comprender el planeta desde el espacio. La teledetección satelital implica la adquisición y análisis de datos obtenidos por sensores montados en satélites que orbitan la Tierra. Estos sensores pueden captar información en diferentes longitudes de onda del espectro electromagnético, lo que permite obtener una visión integral de las condiciones terrestres. Además, la conjunción entre la teledetección satelital, sistemas de información geográfica (SIG) y el impulso que ofrece la inteligencia artificial (IA) ha abierto nuevas posibilidades para el análisis y la interpretación de datos geoespaciales, facilitando la toma de decisiones informadas en la agricultura de precisión.

  En el presente documento, se explorará el uso de información satelital en la agricultura de precisión, destacando sus beneficios, limitaciones y aplicaciones prácticas así como los fundamentos teóricos que sustentan su uso. Se analizarán diferentes tipos de sensores, las técnicas de procesamiento de datos y los índices vegetativos más relevantes para la agricultura. Además de la generación de un modelo de clasificación de cobertura agrícola que pueda ser útil a nivel nacional. Finalmente, se discutirán las perspectivas futuras de esta tecnología y su potencial para transformar la agricultura en los próximos años.

\vspace*{\fill}

\newpage

\subsection{Problematización}

  La agricultura enfrenta numerosos desafíos en la actualidad, incluyendo el cambio climático, la degradación del suelo, la escasez de agua y la necesidad de aumentar la producción para satisfacer la demanda creciente de alimentos. Estos factores afectan directamente el estado de salud de los cultivos y, por ende, la productividad agrícola. La identificación temprana y precisa de problemas en los cultivos es crucial para implementar medidas correctivas que minimicen las pérdidas y optimicen el uso de recursos. Sin embargo, la monitorización tradicional de los cultivos a través de inspecciones visuales y muestreos de campo puede ser limitada en términos de cobertura espacial y temporal, especialmente en grandes extensiones agrícolas. El uso de información satelital en la agricultura de precisión ofrece una alternativa prometedora para superar las limitaciones de los métodos tradicionales, no obstante, es necesario comprender con claridad los fundamentos, alcances y limitaciones de estas tecnologías para aplicarlas adecuadamente en la clasificación de cultivos.

  \begin{itemize}
    \item ¿Cuáles son los fundamentos teóricos que sustentan el uso de información satelital en la agricultura de precisión?
    \item ¿Qué tipos de sensores satelitales y técnicas de procesamiento de datos son más adecuados para la monitorización de cultivos?
    \item ¿Cuáles son los productos derivados de la teledetección satelital más relevantes para la agricultura de precisión?
    \item ¿Cómo se puede aplicar la información satelital para la clasificación de cobertura agrícola en una región específica?
    \item ¿Cuáles son las limitaciones y desafíos asociados con el uso de información satelital en la agricultura de precisión?
    \item ¿Cuáles son las perspectivas futuras para la integración de tecnologías emergentes, como la inteligencia artificial, en la agricultura de precisión basada en información satelital?
  \end{itemize}

\subsection{Justificación}

  La relevancia de este estudio radica en la necesidad de optimizar la producción agrícola mediante el uso de tecnologías avanzadas que permitan una gestión más eficiente y sostenible de los recursos.

  \begin{itemize}
    \item \textbf{Relevancia científica:} La teledetección satelital es una disciplina en constante evolución que integra nuevas tecnologías, metodologías basadas en datos, enfoques interdisciplinarios y avances en sensores, plataformas satelitales, algoritmos de procesamiento y análisis de estos datos y modelos predictivos. Comprender los fundamentos teóricos que sustentan el uso de información satelital en la agricultura de precisión es esencial para aprovechar al máximo su potencial y contribuir al avance del conocimiento en este campo.
    \item \textbf{Relevancia práctica:} La aplicación de información satelital en la agricultura de precisión tiene un impacto directo en la gestión agrícola. Permite a los agricultores tomar decisiones informadas sobre el manejo de cultivos, la aplicación de insumos y la planificación de actividades agrícolas. Esto puede traducirse en una mayor eficiencia en el uso de recursos, reducción de costos y mejora del rendimiento agrícola.
    \item \textbf{Relevancia en la toma de decisiones:} La información satelital proporciona datos en tiempo real que pueden ser utilizados para una mejor política gubernamental en lo que respecta a la toma de decisiones estratégicas en la agricultura. Esto es especialmente relevante en un contexto de cambio climático y variabilidad climática, donde la capacidad de adaptarse rápidamente a las condiciones cambiantes es crucial para la seguridad alimentaria.
  \end{itemize}

\subsection{Objetivos generales y específicos}

  El objetivo general de este estudio es entender y aplicar los fundamentos teóricos del uso de información satelital en la agricultura de precisión para la generación de un modelo de clasificación supervisada de cobertura agrícola en una región específica y luego su uso en distintas regiones. Con ello se pretende contribuir al desarrollo de prácticas agrícolas más eficientes y sostenibles. De este modo se pueden desglosar algunos objetivos específicos que incluyen:

  \begin{itemize}
    \item Explicar los fundamentos teóricos del uso de información satelital en la agricultura de precisión.
    \item Conocer los productos derivados de la teledetección satelital más relevantes para la agricultura de precisión.
    \item Conocer las herramientas tecnológicas disponibles para la clasificación supervisada de cobertura agrícola.
    \item Desarrollar una aplicación práctica de clasificación de cobertura agrícola haciendo uso de alguna de estas herramientas.
    \item Identificar las limitaciones y desafíos asociados con el uso de información satelital en la agricultura de precisión.
    \item Explorar las perspectivas futuras para la integración de tecnologías emergentes, como la inteligencia artificial, en la agricultura de precisión basada en información satelital.
  \end{itemize}

\subsection{Alcances y limitaciones}

  Dentro de este estudio, se establecen ciertos alcances y limitaciones que definen el contexto y la aplicabilidad de los resultados obtenidos.

  \textbf{Alcances:}
  \begin{itemize}
    \item El estudio se centrará en la generación de un modelo de clasificación que pueda ser útil a nivel nacional.
    \item Se utilizarán imágenes satelitales de libre acceso, como lo son las imágenes Sentinel-2, Landsat 8, MODIS y AlphaEarth Foundations.
    \item Se emplearán herramientas tecnológicas disponibles para la clasificación supervisada de cobertura agrícola.
  \end{itemize}

  \textbf{Limitaciones:}
  \begin{itemize}
    \item La disponibilidad y calidad de las imágenes satelitales pueden verse afectadas por condiciones climáticas y otros factores.
    \item La clasificación de cobertura agrícola puede verse limitada por la resolución espacial y temporal de los datos satelitales.
    \item La implementación de tecnologías emergentes, como la inteligencia artificial, puede requerir recursos computacionales significativos y acceso a datos de alta calidad.
  \end{itemize}